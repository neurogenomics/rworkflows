%\VignetteIndexEntry{Bioconductor LaTeX Style 2.0}
%\VignettePackage{BiocStyle}
%\VignetteEngine{utils::Sweave}

\documentclass{article}

\RequirePackage[]{/Library/Frameworks/R.framework/Versions/4.2/Resources/library/BiocStyle/resources/tex/Bioconductor}
\AtBeginDocument{\bibliographystyle{/Library/Frameworks/R.framework/Versions/4.2/Resources/library/BiocStyle/resources/tex/unsrturl}}
\usepackage[noae, nogin]{Sweave}
\usepackage{booktabs} % book-quality tables

\newcommand\BiocStyle{\Rpackage{BiocStyle}}
\newcommand\knitr{\Rpackage{knitr}}
\newcommand\rmarkdown{\Rpackage{rmarkdown}}
\newcommand\Sweave{\software{Sweave}}
\newcommand\latex[1]{{\ttfamily #1}}

\makeatletter
\def\thefpsfigure{\fps@figure}
\makeatother

\newcommand{\exitem}[3]{%
  \item \latex{\textbackslash#1\{#2\}} #3 \csname#1\endcsname{#2}.%
}

\title{\Bioconductor\ \LaTeX\ Style 2.0}
\author[1]{Andrzej Ole\'s}
\author[2]{Martin Morgan}
\author[1]{Wolfgang \mbox{Huber}}
\affil[1]{European Molecular Biology Laboratory, Heidelberg, Germany}
\affil[2]{Roswell Park Cancer Institute, Buffalo, NY}

\usepackage{Sweave}
\begin{document}
\Sconcordance{concordance:LatexStyle2.tex:LatexStyle2.Rnw:%
1 5 1 1 2 1 0 1 2 24 1 1 0 307 1 1 14 144 1 1 2 1 0 2 1 3 0 1 2 77 1 1 %
2 13 0 1 2 3 1 1 11 15 1}


\maketitle

\begin{abstract}
This vignette describes the new Bioconductor \LaTeX\ document style. It aims
to be at once a demonstration of the features of the style and a guide to their use.
The vignette focuses on the use with documents typeset as markup .Rnw files and 
processed using \Sweave\cite{Leisch2002} or \knitr\cite{Xie2014}.
The same style is also applied to \R\ markdown .Rmd files which are rendered to PDF output format. 
%For markdown-specific syntax see the \emph{Bioconductor \R\ markdown style} package vignette.
\end{abstract}

\packageVersion{BiocStyle 2.26.0}

Report issues on \url{https://github.com/Bioconductor/BiocStyle/issues}

\newpage

\tableofcontents

\newpage

\section{Authoring package vignettes}

\subsection{Getting started}

\BiocStyle\ provides standard formatting of documents rendered
to PDF output format. It consists of a \LaTeX\ document style definition which
can be used with either plain \Sweave, \knitr, or \rmarkdown. 
To enable \BiocStyle\ in your package vignette follow the instructions below 
depending on the input format.

\subsubsection{\Sweave}

To use with \Sweave, add the following to your package \file{DESCRIPTION} file:
\begin{verbatim}
    Suggests: BiocStyle
\end{verbatim}
and add this code chunk to the preamble\footnote{before the {\ttfamily \textbackslash begin\{document\}} command, and preferably right after the {\ttfamily \textbackslash documentclass} definition} of your .Rnw file:
\begin{verbatim}
    <<style-Sweave, eval=TRUE, echo=FALSE, results=tex>>=
    BiocStyle::latex()
    @ 
\end{verbatim}

See \Rcode{?latex} for additional options. It is not necessary for Sweave 
vignettes in general to include {\ttfamily\textbackslash usepackage\-\{Sweave\}}
in the document preamble, and it will give you an error if it appears before 
the call to the style function.\footnote{because \BiocStyle\ loads 
{\ttfamily Sweave.sty} in order to override some defaults}
Other \LaTeX\ packages attached by \BiocStyle\ are listed in Appendix~\ref{sec:pkgs}.
Avoid loading these packages yourself, as doing so may cause unexpected side
effects and often leads to document compilation errors.

\subsubsection{\knitr{}}\label{sec:knitr} 

To use with \CRANpkg{knitr}, add the following to the \file{DESCRIPTION} file:
\begin{verbatim}
    VignetteBuilder: knitr
    Suggests: BiocStyle, knitr
\end{verbatim}
this to the top of the .Rnw file:
\begin{verbatim}
    %\VignetteEngine{knitr::knitr}
\end{verbatim}
and this to the preamble:
\begin{verbatim}
    <<style-knitr, eval=TRUE, echo=FALSE, results="asis">>=
    BiocStyle::latex()
    @ 
\end{verbatim}

See \Rcode{?latex} for additional options, and Appendix~\ref{sec:pkgs} for a list of automatically attached \LaTeX\ packages.

\subsubsection{R markdown}

To enable \BiocStyle\ in your \R\ markdown vignette you need to:

\begin{enumerate}
  
  \item Edit to the \file{DESCRIPTION} file by adding

  \begin{verbatim}
    VignetteBuilder: knitr
    Suggests: BiocStyle, knitr, rmarkdown
  \end{verbatim}
    
  \item Specify \Rcode{BiocStyle::pdf\_document} as the output format and add vignette metadata to the document header:

  \begin{verbatim}
    ---
    title: "Vignette Title"
    author: "Vignette Author"
    output: 
      BiocStyle::pdf_document
    vignette: >
      %\VignetteIndexEntry{Vignette Title}
      %\VignetteEngine{knitr::rmarkdown}
      %\VignetteEncoding{UTF-8}  
    ---
  \end{verbatim}
  
\end{enumerate}
  
The \verb+vignette+ section is required in order to instruct \R\ how to build the vignette. Note that \verb+\VignetteIndexEntry+ should match the \verb+title+ of your vignette.
  
See the \emph{\href{http://www.bioconductor.org/packages/devel/bioc/vignettes/BiocStyle/inst/doc/AuthoringRmdVignettes.html}{Authoring R Markdown vignettes}} package vignette for
more details on authoring markdown documents with \BiocStyle.


\subsection{Workflow}

Provided that \BiocStyle\ has been installed, a convenient way to build the
vignette as it is being written is with the command
\begin{verbatim}
    R CMD Sweave --pdf vignette.Rnw
\end{verbatim}
A short-cut useful for checking the \LaTeX{} portion of vignettes is
to toggle evaluation of code chunks to \Rcode{FALSE}.
\begin{verbatim}
    SWEAVE_OPTIONS="eval=FALSE" R CMD Sweave --pdf vignette.Rnw
\end{verbatim}

When using \knitr, the command to process the vignette is
\begin{verbatim}
    R CMD Sweave --engine=knitr::knitr --pdf vignette.Rnw
\end{verbatim}
For \R\ markdown vignettes, set the engine to \verb+knitr::rmarkdown+.
\begin{verbatim}
    R CMD Sweave --engine=knitr::rmarkdown --pdf vignette.Rmd
\end{verbatim}

By default, \knitr\ automatically caches results of vignette
chunks, greatly accelerating the turnaround time required for
edits. Both the default and \knitr\ incantations create PDF
files using \verb+texi2dvi --pdf+; many versions of this software
incorrectly display non-breaking spaces as a tilde, \verb|~|. This can be
remedied (as on the \Bioconductor{} build system) with a final run of
\begin{verbatim}
    R CMD texi2dvi --pdf vignette.tex
    R CMD pdflatex vignette.tex
\end{verbatim}

\section{Structuring the document}

\subsection{Title page}

\warning[Important]{most of the methods described here only work once \BiocStyle\ is loaded;
therefore, specify title and authors \emph{after} the code chunk calling \Rfunction{BiocStyle::latex()}.}

\subsubsection{Title and running headers}

Create a title and running headers by defining \verb+\bioctitle+
and \verb+\author+ in the preamble
\begin{verbatim}
    \bioctitle[Short title for headers]{Full title for title page}
    %% also: \bioctitle{Title used for both header and title page} 
    %% or... \title{Title used for both header and title page} 
    \author{Vignette Author\thanks{\email{user@domain.com}}}
\end{verbatim}
and use \verb+\maketitle+ at the beginning of the document to output the title and author information.

\subsubsection{Authors and affiliations}

As illustrated above, use the \verb+\email+ command to enter hyperlinked email addresses
typeset in monospace font. Multiple author affiliations can be specified with the help of
the \LaTeX\ package \latex{authblk} which is automatically loaded by \BiocStyle.
See the following examples for typical use cases. 

If your vignette has just a single author, use standard \LaTeX\ syntax
to enter the author and affiliation information separated by a new line.
You can provide the email address in \verb+\thanks+, or in another new line 
after the affiliation.

\begin{verbatim}
    \author{Vignette Author\thanks{\email{user@domain.com}}\\
            Author's Affiliation}
\end{verbatim}

In case of multiple authors, you can specify the author and affiliation information in different manners depending on 
the number of affiliations. In the basic case, when you don't provide authors'
affiliation, or all the authors share the same affiliation, you can just use the regular \verb+\author+ command:%
\footnote{the usual author separator {\ttfamily \textbackslash and} won't work because the affiliations are typeset in footnote mode}

\begin{verbatim}
    \author{
      First Author\thanks{\email{first@author.com}} and
      Second Author\thanks{\email{second@author.com}}\\
      Shared Affiliation
    }
\end{verbatim}

To provide different affiliations, some of which are potentially shared, 
enter each author separately and use \verb+\affil+
to specify affiliation. The authors will appear all in one (possibly continued) line
with automatic footnotes, and the affiliations are displayed in separate lines below.\footnote{as on this vignette's title page}

\begin{verbatim}
    \author{First Author}
    \author{Second Author}
    \affil{First and Second authors' shared affiliation}
    \author{Third Author\thanks{\email{corresponding@author.com}}}
    \affil{Third author's affiliation}
\end{verbatim}

In the case when some authors have more than one affiliation, or the authors with 
a shared affiliation do not come one after another, you need to manually define
footnote markers as optional arguments to \verb+\author+ and \verb+\affil+ commands, as in the following example.

\begin{verbatim}
    \author[1,2]{First Author\thanks{\email{user@domain.com}}}
    \author[1]{Second Author}
    \author[2]{Third Author}
    \affil[1]{First and Second authors' affiliation}
    \affil[2]{First and Third authors' affiliation}
\end{verbatim}


\subsubsection{Abstract, package version, and table of contents}

Abstract can be entered using the standard \verb+abstract+ environment:

\begin{verbatim}
    \begin{abstract}
      Short summary of the vignette's content.
    \end{abstract}
\end{verbatim}

It is recommended to include information on the specific 
package version described in the vignette. The following line\footnote{substitute 
\latex{packageName} by the name of your package} inserts
this information automatically and in a properly formatted manner.

\verb|    \packageVersion{\|\verb|Sexpr{BiocStyle::pkg_ver("packageName")}}|

Use \verb+\tableofcontents+ to include a hyperlinked table of contents (TOC), and
\verb+\section+, \verb+\subsection+, \verb+\subsubsection+ for
structuring your document.  It is a good practice to start the first section
on a new page by calling \verb+\newpage+ after the TOC:

\begin{verbatim}
    \tableofcontents
    \newpage
    \section{My First Section}
\end{verbatim}

\subsection{Style macros}

\BiocStyle\ introduces the following additional markup styling commands
useful in typical \Bioconductor{} vignettes.\\\\
%%
Software:
\begin{itemize}
  \item \verb+\R{}+ and \verb+\Bioconductor{}+ to reference \R{} software and
  the \Bioconductor{} project.
  \exitem{software}{GATK}{to reference third-party software, e.g.,}
\end{itemize}
%\vspace{1em}
%%
Packages:
\begin{itemize}
  \exitem{Biocpkg}{IRanges}{for \Bioconductor{} software, annotation and experiment data packages, including a
  link to the release landing page or if the package is only in devel, to the devel landing page.}
  \exitem{CRANpkg}{data.table}{for \R{} packages available on CRAN, including
  a link to the FHCRC CRAN mirror landing page,}
  \exitem{Githubpkg}{rstudio/rmarkdown}{for \R{} packages available on GitHub, including a link to the package repository,}
  \exitem{Rpackage}{MyPkg}{for \R{} packages that are \emph{not} available on
  \Bioconductor{} or CRAN,}
\end{itemize}
%\vspace{1em}
%%
Code:
\begin{itemize}
  \exitem{Rfunction}{findOverlaps}{for functions}
  \exitem{Robject}{olaps}{for variables}
  \exitem{Rclass}{GRanges}{when referring to a formal class}
  \exitem{Rcode}{log(x)}{for \R{} code,}
\end{itemize}
%\vspace{1em}
%%
Communication:
\begin{itemize}
  \exitem{bioccomment}{additional information for the user}{communicates}
  \exitem{warning}{common pitfalls}{signals}
  \exitem{fixme}{incomplete functionality}{provides an indication of}
\end{itemize}
For all of the above message types, the default opening word can be overriden in an optional argument, e.g.~\latex{\textbackslash{fixme}[TODO]\{missing functionality\}} produces \fixme[TODO]{missing functionality}.

General:
\begin{itemize}
  \exitem{email}{user@domain.com}{to provide a linked email address,}
  \exitem{file}{script.R}{for file names and file paths}
\end{itemize}

\subsection{Code chunks}

%\BiocStyle\ changes the default setting of the \verb+error+ option

\BiocStyle\ uses some heuristics to automatically adjust the line length\footnote{\Rcode{options("width")}} 
of output code chunks such that it matches the document font size setting.\footnote{for the default font size of 10pt the code width is 80 columns}
If for some reason you wish to alter it, use the \Rcode{width} argument in the call to \Rfunction{BiocStyle::latex}.

%---------------------------------------------------------
\subsection{Figures}\label{sec:figures}
%---------------------------------------------------------


In addition to the standard \verb+\figure+ environment \BiocStyle\ provides special
environments for small and wide figures. These three environments differ in size, 
default aspect ratio\footnote{device dimensions are automatically set only by \knitr\ and \rmarkdown, and not by \Sweave}, 
and horizontal placement on the the page. See the following
table for details, and Figures~\ref{fig:figure}, \ref{fig:wide} and \ref{fig:small} 
for examples.

\begin{table}
  \begin{small}
  \begin{tabular}{llp{6.5mm}lll}
    \toprule
    environment & description & \multicolumn{2}{l}{output width} & device dimensions & aspect ratio\\
    \midrule
    \latex{figure} & regular figure & 120mm &($\approx$4.72in) & 8 x 5 in & 1.6\\
    \latex{figure*} & wide figure & 150mm  &($\approx$5.90in)& 10 x 5 in & 2.0\\
    \latex{smallfigure} & small figure & 75mm &($\approx$2.95in)& 5 x 5 in & 1.0 \\
    \bottomrule
  \end{tabular}
  \end{small}
  \label{tab:figs}
\end{table}

\begin{figure}
  \includegraphics{\jobname-fig}
  \caption{\label{fig:figure}Regular figure. A plot displayed with the \latex{\textbackslash figure} environment.}
\end{figure}

\begin{figure*}
  \includegraphics{\jobname-widefig}
  \caption{\label{fig:wide}Wide figure. A plot displayed
  with the {\ttfamily\textbackslash figure*} environment.}
\end{figure*}

\begin{smallfigure}
  \includegraphics{\jobname-smallfig}
  \caption{\label{fig:small}Small figure. A plot displayed with the \latex{\textbackslash{}smallfigure} environment.}
\end{smallfigure}

The \verb+figure+ environment produces regular figures which are left-indented
and right-aligned with the paragraph, as Figure~\ref{fig:figure}.
To utilize the whole available width use the \verb+figure*+ environment.
It produces figures which are left-aligned with the paragraph
and extend on the right margin (Figure~\ref{fig:wide}).
The \verb+smallfigure+ environment is meant for possibly rectangular plots
which are about half as wide as the paragraph (Figure~\ref{fig:small}).
The default placement specifier for \BiocStyle\ floats is \latex{\thefpsfigure}, 
which typically outputs them in the place where they are defined. 
The first sentence of figure captions is emphasized to serve as figure title. 
This feature can be disabled be setting the argument \Rcode{titlecaps=FALSE} in the call to \Rfunction{BiocStyle::latex}.

To use figure environments in \Sweave, write explicit \LaTeX\ code which inserts
them in combination with the \Sweave\ option \verb+include=FALSE+. 
For example, Figure~\ref{fig:wide} was produced with the following code.

\begin{verbatim}
    <<widefig, echo=FALSE, fig=TRUE, include=FALSE, width=10, height=5>>=
    par(mar=c(4,4,0.5,0.5))
    plot(cars)
    @
\end{verbatim}
\begin{verbatim}
    \begin{figure*}
      \includegraphics{\jobname-widefig}
      \caption{\label{fig:wide}Wide figure. A plot displayed
      with the {\ttfamily\textbackslash figure*} environment.}
    \end{figure*}
\end{verbatim}

In \knitr\ and \rmarkdown\ the output environment for a graphics-producing code
chunk can be specified in \Rcode{fig.env} chunk option, e.g.~set 
\Rcode{fig.env='smallfigure'} to get \verb+\begin{smallfigure}+. 
It is also possible to specify the wide and small figure
environments by setting \Rcode{fig.wide} or \Rcode{fig.small} option to \Rcode{TRUE}.
The following two \knitr\ code chunks are equivalent and produce the same output, similar to Figure~\ref{fig:small}.
\begin{verbatim}
    <<smallfig, fig.cap="Small figure.", fig.env="smallfigure">>=
    plot(cars)
    @
\end{verbatim}
\begin{verbatim}
    <<smallfig, fig.cap="Small figure.", fig.small=TRUE>>=
    plot(cars)
    @
\end{verbatim}

Specify \Rcode{fig.width} and/or \Rcode{fig.height} to override the default 
device dimensions listed in the table on page~\pageref{tab:figs}. To adjust the 
aspect ratio use \Rcode{fig.asp}. For example, the following code would produce 
a full-width square plot, as in Figure~\ref{fig:square}.

\begin{verbatim}
    <<square, fig.wide=TRUE, fig.asp=1>>=
    plot(cars)
    @
\end{verbatim}

\begin{figure*}
  \includegraphics{\jobname-squarefig}
  \caption{Full-width square figure. Use {\ttfamily fig.wide=TRUE, fig.asp=1} code chunk options to enter with \knitr\ or \rmarkdown.\label{fig:square}}
\end{figure*}

When \knitr\ is used plots are cropped so that blank margins around them are removed to make better use of the space in the output document\footnote{otherwise one often has to struggle with \Rcode{par} to set appropriate margins}. This feature can be switched off by setting the chunk option \Rcode{knitr::opts\_chunk\$set(crop = NULL)}.

\subsubsection{\latex{\textbackslash incfig} convenience macro}\label{incfig}

Besides the usual \LaTeX\ capabilities (the \verb+figure+ environment and
\verb+\includegraphics+ command) \BiocStyle\ defines a macro 

\verb+    \incfig[placement]{filename}{width}{title}{caption}+

which expects four arguments:
\begin{description}[font=\ttfamily]
\item[filename] The name of the figure file, also used as the label by
  which the float can be referred to by \verb+\ref{}+. Some
  \Sweave\ and \knitr\ options place figures in a
  subdirectory; unless \Rcode{short.fignames=TRUE} is set the full file name,
  including the subdirectory and any prefixes, should be provided. By default,
  these are \file{<sweavename>-} for \Sweave\ and \file{figure/} for
  \knitr. Also note the different naming scheme used by
  \knitr: figure files are named \file{<chunkname>-i} where \emph{i}
  is the number of the plot generated in the chunk. 
\item[width] Figure width.
\item[title] A short title, used in the list of figures and
  printed in bold as the first part of the caption.
\item[caption] Extended description of the figure.
\end{description}
The optional \textbf{\ttfamily placement} specifier controls where the figure is placed
on page; it takes the usual values allowed by \LaTeX{} floats, i.e., a list
containing \verb+t+, \verb+b+, \verb+p+, or \verb+h+, where letters enumerate
permitted placements\footnote{if no placement specifier is given, the default
\latex{\thefpsfigure} is assumed}.

For \verb+incfig+ with \Sweave\, use
\begin{verbatim}
    <<figureexample, fig=TRUE, include=FALSE, width=4.2, height=4.6>>=
    par(mar=c(4,4,0.5,0.5))
    v = seq(0, 60i, length=1000)
    plot(abs(v)*exp(v), type="l", col="Royalblue")
    @
    \incfig{\jobname-figureexample}{0.5\textwidth}{A curve.}
      {The code that creates this figure is shown in the code chunk.}
    as shown in Figure~\ref{\jobname-figureexample}.
\end{verbatim}
This results in
\begin{Schunk}
\begin{Sinput}
> par(mar=c(4,4,0.5,0.5))
> v = seq(0, 60i, length=1000)
> plot(abs(v)*exp(v), type="l", col="Royalblue")
\end{Sinput}
\end{Schunk}
\incfig{\jobname-figureexample}{0.5\textwidth}{A curve.}
  {The code that creates this figure is shown in the code chunk.}
as shown in Figure~\ref{\jobname-figureexample}.  

When option \Rcode{short.fignames=TRUE} is set, figure names used by
\verb+\incfig+ and \verb+\ref+ do not contain any prefix and are
identical to the corresponding code chunk labels (plus figure number in case of
\knitr). For example, in \Sweave\ the respective code for the above
example would be \verb+\incfig{figureexample}{...}{...}{...}+ and
\verb+\ref{figureexample}+.

For \verb+\incfig+ with \knitr, use the option
\Rcode{fig.show='hide'} rather than \Rcode{include=FALSE}. The
\knitr-equivalent code for
Figure~\ref{LatexStyle2-figureexample} is:
\begin{verbatim}
    <<figureexample, fig.show='hide', fig.width=4, fig.height=4>>=
    par(mar=c(4,4,0.5,0.5))
    v = seq(0, 60i, length=1000)
    plot(abs(v)*exp(v), type="l", col="Royalblue")
    @
\end{verbatim}
Note the difference in option names setting the figure width and
height compared to \Sweave.  Unless
\Rcode{short.fignames=TRUE} is set, use the default \file{figure/}
prefix when inserting and referring to figures, e.g.:
\begin{verbatim}
    \incfig{figure/figureexample-1}{0.5\textwidth}{A curve.}
      {The code that creates this figure is shown in the code chunk.}
\end{verbatim}
A custom prefix for figure file names can be passed to
\Rfunction{latex} in the \Rcode{fig.path} argument. When
\Rcode{short.fignames=TRUE}, figures can be referred to directly by
code chunk labels, i.e., \verb+\incfig{figureexample-1}...+ and
\verb+\ref{figureexample-1}+.

\subsection{Equations}

When referencing equations, e.g.~\eqref{equation}, use \verb+\eqref+ to ensure proper label formatting.
%
\begin{equation}\label{equation}
  \sin^2 \theta + \cos^2 \theta \equiv 1
\end{equation}

\subsection{Footnotes}

One of the distinctive features of the style is an asymmetric page layout 
with a wide margin on the right. This provides additional space for ancillary information in side notes. These can be entered in footnotes\footnote{this is a side note entered using the {\ttfamily \textbackslash footnote} command} typeset as margin notes, which has the advantage that the notes appear close to the place where they are defined. 

%---------------------------------------------------------
\subsection{Bibliography}
%---------------------------------------------------------

\Rfunction{BiocStyle::latex} has default argument
\Rcode{use.unsrturl=TRUE} to automatically format bibliographies using
\software{natbib}'s unsrturl style. There is no need to explicitly
include \software{natbib}, and it is an error to use a
\verb+\bibliographystyle+ command. The \software{unsrturl.bst} format,
e.g., \cite{Leisch2002, Xie2014}, supports
hyperlinks to DOI and PubMed IDs but not \verb+\citet+ or
\verb+\citep+.

To use a bibliography style different from \verb+unsrturl+, set
\Rcode{use.unsrturl=FALSE} and follow normal \LaTeX{} conventions.


\bibliography{Bioc}

\newpage

\appendix

%---------------------------------------------------------
\section{Session info}
%---------------------------------------------------------

Here is the output of \Rfunction{sessionInfo()} on the system on which
this document was compiled:
\begin{itemize}\raggedright
  \item R version 4.2.1 (2022-06-23), \verb|x86_64-apple-darwin17.0|
  \item Locale: \verb|en_US.UTF-8/en_US.UTF-8/en_US.UTF-8/C/en_US.UTF-8/en_US.UTF-8|
  \item Running under: \verb|macOS Big Sur ... 10.16|
  \item Matrix products: default
  \item BLAS:   \verb|/Library/Frameworks/R.framework/Versions/4.2/Resources/lib/libRblas.0.dylib|
  \item LAPACK: \verb|/Library/Frameworks/R.framework/Versions/4.2/Resources/lib/libRlapack.dylib|
  \item Base packages: base, datasets, graphics, grDevices, methods,
    stats, utils
  \item Loaded via a namespace (and not attached): BiocManager~1.30.20,
    BiocStyle~2.26.0, cli~3.6.1, compiler~4.2.1, digest~0.6.31,
    evaluate~0.21, fastmap~1.1.1, htmltools~0.5.5, knitr~1.43,
    rlang~1.1.1, rmarkdown~2.22, rstudioapi~0.14, tools~4.2.1,
    xfun~0.39, yaml~2.3.7
\end{itemize}

\section{Attached \LaTeX\ packages}\label{sec:pkgs}


\BiocStyle\ loads the following \LaTeX\ packages:

\latex{authblk}, \latex{beramono}, \latex{caption}, \latex{changepage}, \latex{color}, \latex{enumitem}, \latex{etoolbox}, \latex{fancyhdr}, \latex{fontenc}, \latex{footmisc}, \latex{framed}, \latex{geometry}, \latex{graphicx}, \latex{hyperref}, \latex{ifthen}, \latex{marginfix}, \latex{mathtools}, \latex{nowidow}, \latex{parnotes}, \latex{parskip}, \latex{placeins}, \latex{ragged2e}, \latex{soul}, \latex{titlesec}, \latex{titletoc}, \latex{xstring}.


\section{Known issues}

\subsection{Compatibility with the \Rpackage{xtable} \R\ package}

\BiocStyle\ does not support tables produced by the \R\ package \CRANpkg{xtable} in plain \Sweave\ documents.
This limitation does not apply to documents compiled with \knitr. If you would like to 
continue using \Rpackage{xtable} in your \BiocStyle-enabled \Sweave\ document, please consider migrating it to \knitr. See \Rcode{?knitr::Sweave2knitr} for automatic conversion, and Sections~\ref{sec:knitr} and \ref{sec:figures} for details on using \BiocStyle\ with \knitr.


\end{document}
